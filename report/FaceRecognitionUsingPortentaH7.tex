%%%%%%
%
% $Autor: Wings $
% $Datum: 2020-01-18 11:15:45Z $
% $Pfad: githubtemplate/Template/report/rename.tex $
% $Version: 4620 $
%
%
% !TeX encoding = utf8
% !TeX root = Rename
% !TeX TXS-program:bibliography = txs:///bibtex
%
%%%%%%


\documentclass[12pt,a4paper]{scrbook}


% Auswahl der Sprache/Language selection
% Die nicht gewünschte Sprache muss auskommentiert werden:/The language that is not desired must be commented out:
%\def\isGerman{1}
\def\isEnglish{1}

%\DeclareUnicodeCharacter{03A9}{\ensuremath{\Omega}}


\chapter{Packages}
\section{edge-impulse-sdk, Version 1.3.0}
\subsection{Introduction}
The \SHELL{edge-impulse-sdk} is a versatile library designed to run machine learning inferences on edge devices like the Portenta H7. In our face recognition project, this SDK allows us to deploy the model trained using Edge Impulse, simplifying the integration of AI into our system for real-time facial recognition tasks. \cite{edgeimpulse_cpp_library:2024}

\subsection{Installation}

To install the ‘edge-impulse-sdk‘, follow these steps:
\begin{itemize}
	\item Download the C++ library from the Edge Impulse Studio.
	\item Unzip the library into your project directory.
	\item Ensure you have a C compiler, a C++ compiler, and Make installed on your system.
\end{itemize}

For detailed instructions, refer to the \href{https://docs.edgeimpulse.com/doc}{Edge Impulse Documentation}.

\subsection{Example - Description}
This example demonstrates how to set up and use the ‘edge-impulse-sdk‘ to perform inference on raw sensor data. The setup involves creating a C++ project, integrating the SDK, and running the impulse to get the model’s output.


\subsection{Example - Manual}

\begin{enumerate}
	\item \textbf{Create a Project:} Set up a directory for your project and unzip the C++ library.
	
	\item \textbf{Write Code:} Create a main application file (\texttt{main.cpp}) that includes the necessary headers and defines the main function to run the classifier.
	
	\item \textbf{Build the Project:} Use \texttt{Make} to compile the project and generate the executable.
	
	\item \textbf{Run Inference:} Execute the program to run inference on the test data.
\end{enumerate}

\subsection{Example - code}
Here is a simple example ~\ref{ExampleCode_EdgeImpulseSDK.ino} of a main.cpp file:


{
	\captionof{code}{Package exmaple code}\label{ExampleCode_EdgeImpulseSDK.ino}
	\ArduinoExternal{}{../Code/EdgeImpulse/Packages/ExampleCodeSDK.ino}
}


\subsection{Future Reading}
For more information on deploying your model as a C++ library, check the \href{https://docs.edgeimpulse.com/docs/run-inference/cpp-library/deploy-your-model-as-a-c-library}{Edge Impulse Documentation}

\newpage
\section{TensorFlow Lite, Version 2.12.0}
\subsection{Introduction}
TensorFlow Lite is a lightweight solution for deploying machine learning models on mobile and embedded devices. It is designed to run TensorFlow models on resource-constrained devices with low latency and high efficiency. \cite{tensorflowLiteGuide:2024}


\subsection{Installation}
To install TensorFlow Lite, run \SHELL{pip install tensorflow} in command prompt window, refer ~\ref{tf_install}

\subsection{Example - Description}
This example demonstrates how to use TensorFlow Lite to perform image classification on a mobile device. The steps include loading a model, preprocessing input data, running inference, and interpreting the results.

\subsection{Example - Manual}
The manual steps for running a TensorFlow Lite model include:
\begin{enumerate}
	\item Convert the TensorFlow model to TensorFlow Lite format.
	\item Load the TensorFlow Lite model on the device.
	\item Preprocess the input data to match the model’s requirements.
	\item Use the TensorFlow Lite interpreter to run the model.
	\item Postprocess the results to interpret the model’s output.
\end{enumerate}

\subsection{Example - Code}

{
	\captionof{code}{Example Code for running Tensorflow Lite model.}\label{ExampleCode_TensorFlow.ino}
	\ArduinoExternal{}{../Code/EdgeImpulse/Packages/ExampleCodeTensorFlow.ino}
}



	
\subsection{Example - Files}
The example files include:
\begin{itemize}
	\item \texttt{model.tflite} - The TensorFlow Lite model file.
	\item \texttt{image.jpg} - The input image file for inference.
	\item \texttt{tflite\_example.py} - The Python script to run the model.
\end{itemize}

\subsection{Futher reading}
For more detailed information, refer to the TensorFlow Lite documentation:
\begin{itemize}
		\item \href{https://www.tensorflow.org/lite/guide}{TensorFlow Lite Guide}.
		\item \href{https://www.tensorflow.org/lite/models}{TensorFlow Lite Models}.
		\item \href{https://www.tensorflow.org/lite/convert}{TensorFlow Lite Conversion}.
	
\end{itemize}



\include{General/commands}
\include{General/tikzdefs}

\input{General/acronyms}
\input{General/Hyphenations}%

\usepackage{tikz}
\usetikzlibrary{arrows.meta, positioning}
\usetikzlibrary{shapes.geometric, arrows}



\addbibresource{../Documents/literature.bib}
\addbibresource{../Documents/MyLiterature.bib}

\usepackage{listings}
\usepackage{xcolor}
\usepackage{comment}

\usepackage{array}

\usepackage{hyperref}
\usepackage{graphicx} % for handling graphics
\usepackage{longtable}
\usepackage{float}

\usepackage{caption}

% Configure listings
\lstset{
	basicstyle=\ttfamily,
	breaklines=true,
	numbers=left,
	numberstyle=\tiny,
	stepnumber=1,
	numbersep=5pt,
	frame=single,
}

% Define the colors
\definecolor{myblue}{RGB}{0, 100, 204}  % Blue color for the YAML syntax
\definecolor{mybrown}{RGB}{165, 42, 42} % Brown color for text after colon

\lstdefinelanguage{yaml}{
	basicstyle=\ttfamily\color{myblue},  % Set all text in blue
	morestring=[b]",
	sensitive=false,
	morecomment=[l]{\#},
	commentstyle=\color{gray},
	stringstyle=\color{mybrown},  % Default color for strings after colon
	showstringspaces=false,
	breaklines=true,
	frame=single,
	tabsize=2,
	moredelim=[l][\color{mybrown}]{:},  % Color text after the first colon in brown
	literate=
	{true}{{\color{myblue}true}}1
	{false}{{\color{myblue}false}}1
}
% Set up listings style
\lstset{
	language=C,
	basicstyle=\ttfamily\small,
	keywordstyle=\color{blue},
	commentstyle=\color{green},
	stringstyle=\color{red},
	showstringspaces=false,
	breaklines=true,
	frame=lines,
	captionpos=b
}

% Define Arduino language settings for listings
\lstdefinelanguage{Arduino}{
	morekeywords={setup, loop, pinMode, digitalWrite, HIGH, LOW, delay},
	morecomment=[l]{//},
	morecomment=[s]{/*}{*/},
	morestring=[b]",
	sensitive=true
}

% Define C++ language and style
\lstset{ %
	language=C++,                % choose the language of the code
	basicstyle=\ttfamily\small,  % the size of the fonts that are used for the code
	numbers=left,                % where to put the line-numbers
	numberstyle=\tiny\color{gray}, % the style that is used for the line-numbers
	stepnumber=1,                % the step between two line-numbers. If it's 1, each line will be numbered
	numbersep=5pt,               % how far the line-numbers are from the code
	backgroundcolor=\color{white}, % choose the background color. You must add \usepackage{color}
	showspaces=false,            % show spaces adding particular underscores
	showstringspaces=false,      % underline spaces within strings
	showtabs=false,              % show tabs within strings adding particular underscores
	frame=single,                % adds a frame around the code
	rulecolor=\color{black},     % if not set, the frame-color may be changed on line-breaks within not-black text
	tabsize=2,                   % sets default tabsize to 2 spaces
	captionpos=b,                % sets the caption-position to bottom
	breaklines=true,             % sets automatic line breaking
	breakatwhitespace=false,     % sets if automatic breaks should only happen at whitespace
	title=\lstname,              % show the filename of files included with \lstinputlisting
	keywordstyle=\color{blue},   % keyword style
	commentstyle=\color{green},  % comment style
	stringstyle=\color{red},     % string literal style
}

% Set up the appearance of code listings
\lstset{
	language=Arduino,
	basicstyle=\ttfamily\small, % Code font and size
	keywordstyle=\color{blue}\bfseries, % Keywords
	stringstyle=\color{red}, % Strings
	commentstyle=\color{green}\itshape, % Comments
	showstringspaces=false, % Do not show spaces in strings
	numberstyle=\tiny\color{gray}, % Line numbers
	numbersep=5pt,
	frame=single, % Frame around code
	captionpos=b, % Caption position
	breaklines=true, % Automatic line breaking
	breakatwhitespace=true % Break at whitespace
}


\lstset{
	basicstyle=\ttfamily,
	numbers=left,
	numberstyle=\tiny,
	stepnumber=1,
	numbersep=5pt,
	backgroundcolor=\color{white},
	showspaces=false,
	showstringspaces=false,
	showtabs=false,
	frame=single,
	rulecolor=\color{black},
	tabsize=2,
	captionpos=b,
	breaklines=true,
	breakatwhitespace=false,
	title=\lstname,
	keywordstyle=\color{blue},
	commentstyle=\color{green},
	stringstyle=\color{red},
}


\begin{document}
 

\InputLanguage{Contents/}{Titlepage} 
\newpage

\pagenumbering{roman}

\tableofcontents
\cleardoublepage

\listoffigures
\cleardoublepage


\listoftables
\cleardoublepage


%\addcontentsline{toc}{chapter}{\TRANS{Liste der Programme}{List of Listings}}
%\lstlistoflistings
\listofcodes
\cleardoublepage

% Liste alle Listings aus Dateien
%\addcontentsline{toc}{chapter}{\TRANS{Liste der Programme}{List of Listings}}
%\lstlistoflistings
%\cleardoublepage



%\addcontentsline{toc}{chapter}{Acronyms}
%\printacronyms
%\cleardoublepage
 

\InputLanguage{Contents/}{Symbols}

\printnomenclature
\cleardoublepage


 
\pagenumbering{arabic}

\part{Introduction}

%%%%%%%%%%%%%
%
% $Autor: Wings $
% $Datum: 2019-03-05 08:03:15Z $
% $Pfad: TemplateSensor $
% $Version: 4250 $
% !TeX spellcheck = en_GB/de_DE
% !TeX encoding = utf8
% !TeX root = filename 
% !TeX TXS-program:bibliography = txs:///biber
%
%%%%%%%%%%%%

% Structure
\chapter{Sensor}

Introduction
\Mynote{cite books, applications, board}

\section{General}

General description

cite books

\section{Specific Sensor}

cite board

\section{Specification}

\begin{itemize}
  \item cite data sheet
  \item Circuit Diagram
\end{itemize}

\section{Library}

\subsection{Description}

\subsection{Installation}

\subsection{Functions}

\subsection{Example's Manual}

\subsection{Inside the Example}

\subsection{Example's Code}

\subsection{Example's Files}



\section{Calibration}

cite method

\section{Simple Code}


\section{Simple Application}



\section{Tests}

\subsection{Simple Function Test}

\subsection{Test all Functions}

\section{Simple Application}


\section{Further Readings}



%\input{Contents/Templates/Package}

\newpage
\InputLanguage{Contents/}{Introduction}

\part{Arduino Development Tools}

\newpage
\InputLanguage{Contents/}{Arduino}

%\newpage
%\InputLanguage{Contents/}{FirstStepwiththePortentaH7}

\newpage
\InputLanguage{Contents/}{ArduinoWebEditor}

\newpage
\InputLanguage{Contents/}{ArduinoCLI}

\part{Portenta H7}
\newpage
\InputLanguage{Contents/}{ArduinoPortentaH7}

\newpage
\InputLanguage{Contents/}{ArduinoPortentaH72}

\newpage
\InputLanguage{Contents/}{FirstStepwiththePortentaH7}

%\newpage
%\InputLanguage{Contents/}{FirstStepwithPortentaH7}

%\newpage
%\InputLanguage{Contents/}{PortentaH7}

\newpage
\InputLanguage{Contents/}{PortentaVisionShield}

\newpage
\InputLanguage{Contents/}{FirstStepVisionShield}

\newpage
\InputLanguage{Contents/}{PortentaIoTGNSS}

%\newpage
%\InputLanguage{Contents/}{VisionShield}

%\newpage
%\InputLanguage{Contents/}{PortentaH7Hardware} 

\part{Arduino PortentaH7 - Onboard Sensors}

\newpage
\InputLanguage{Contents/}{PowerLEDDescription}

\newpage
\InputLanguage{Contents/}{LSM6DSO}

\newpage
\InputLanguage{Contents/}{CodeEx}

\newpage
\InputLanguage{Contents/}{BluetoothDescription}

\newpage
\InputLanguage{Contents/}{WIFIDescription}

\newpage
\InputLanguage{Contents/}{LORA}

%\newpage
%\InputLanguage{Contents/}{PushButton}	

%\newpage
%\InputLanguage{Contents/}{IMUDescription}	



%\newpage
%\InputLanguage{Contents/}{VisionShield}

\part{Tools and Packages}
\newpage
\InputLanguage{Contents/}{TensorFlowLite}

\part{Domain Machine Learning}

\newpage
\InputLanguage{Contents/}{Algorithm}

\newpage
\InputLanguage{Contents/}{Packages}

\part{Methodology}
\newpage
\InputLanguage{Contents/}{Methodology}

\part{ KDD}
\newpage
\InputLanguage{Contents/}{FaceAccess}

\part{Appendix}
\chapter{Application Appendix}

\section{List of Materials}
Please refer to Table~\ref{tab:material-list}.

% Table 27.1: Material List

\begin{table}[h!]
	\centering
	\caption{Material List}
	\resizebox{\textwidth}{!}{%
		\begin{tabular}{|l|l|l|l|l|}
			\hline
			\textbf{Title}          & \textbf{Version} & \textbf{Description}                     & \textbf{Price} & \textbf{Source} \\ \hline
			Laptop                  & Windows 11       & Lenovo 15.6 Inch Full HD Notebook         & \$366          & \href{https://www.amazon.com}{Amazon} \\ \hline
			Portenta H7 Board       & Latest           & Development Board with Wi-Fi and LoRa support & \$130          & \href{https://www.arduino.cc}{Arduino} \\ \hline
			Vision Shield           & OV5640 Camera + LoRa & Camera module for Portenta H7            & \$65           & \href{https://www.arduino.cc}{Arduino} \\ \hline
			USB Cable               & USB C            & Cable for connecting Portenta H7 to laptop & \$3            & \href{https://www.amazon.com}{Amazon} \\ \hline
			Micro SD Card           & 8 GB             & Storage for Vision Shield                 & \$5            & \href{https://www.amazon.com}{Amazon} \\ \hline
		\end{tabular}%
	}
	\label{tab:material-list}
\end{table}


\section{Software Bill of Materials}
Please refer to Table~\ref{tab:software-bill}.

\begin{table}[h!]
	\centering
	\caption{Software Bill of Materials}
	\begin{tabular}{|l|l|l|l|}
		\hline
		\textbf{Title} & \textbf{Version} & \textbf{Description} & \textbf{Source} \\ \hline
		Arduino IDE & 2.3.2 & IDE for Arduino boards & \href{https://www.arduino.cc}{arduino.cc} \\ \hline
		Edge Impulse CLI & 1.15.4 & CLI for Edge Impulse models & \href{https://www.edgeimpulse.com}{edgeimpulse.com} \\ \hline
		Python & 3.10 & Python programming language & \href{https://www.python.org}{python.org} \\ \hline
	\end{tabular}
	\label{tab:software-bill}
\end{table}

\section{List of Packages}
Please refer to Table~\ref{tab:package-list}.

% Table 27.3: List of Packages
\begin{table}[h!]
	\centering
	\caption{List of Packages}
	\resizebox{\textwidth}{!}{%
		\begin{tabular}{|l|l|l|l|l|}
			\hline
			\textbf{Title} & \textbf{Version} & \textbf{Price} & \textbf{Description} & \textbf{Source} \\ \hline
			TensorFlow Lite & 2.12.0 & Free & Lightweight version of TensorFlow for embedded devices & \href{https://www.tensorflow.org/lite}{tensorflow.org} \\ \hline
			edge-impulse-sdk & 1.3.0 & Free & SDK for deploying Edge Impulse models & \href{https://www.edgeimpulse.com}{edgeimpulse.com} \\ \hline
			arduino-libraries & Latest & Free & Arduino libraries for Vision Shield and LoRa & \href{https://github.com/arduino}{github.com/arduino} \\ \hline
		\end{tabular}%
	}
	\label{tab:package-list}
\end{table}

\section{Requirement File}
Here is the path to the requirements file, \texttt{requirements.txt}:
\begin{verbatim}
	Project/requirements.txt
\end{verbatim}

The content of the requirements file is:
\begin{verbatim}
	# requirements.txt
	tensorflow-lite==2.12.0
	numpy==1.24.3
	edge-impulse-cli==1.15.4
\end{verbatim}

\section{Programming Languages}
Please refer to Table~\ref{tab:programming-languages}.

\begin{table}[h!]
	\centering
	\caption{Programming Languages}
	\begin{tabular}{|l|l|}
		\hline
		\textbf{Language} & \textbf{Version} \\ \hline
		Python & 3.10 \\ \hline
		C++ & Latest \\ \hline
	\end{tabular}
	\label{tab:programming-languages}
\end{table}


\begin{comment}
\chapter{Methodology}

Domain Knowledge:

\begin{itemize}
	\item Part Domain, e.g. HW, application
	\item Part technologies, e.g. algorthms
	\item Part Tools, e.g. IDE
\end{itemize}
\end{comment}



\chapter{doxygen	 - Example}

Doxygen is an open-source documentation generator that creates detailed and structured documentation from annotated source code. It supports various programming languages such as C, C++, Python, and Java.	

In this project, Doxygen was utilized to document the source code and provide a clear explanation of the functionality of each module, class, and function as shown in the figure ~\ref{Doxygen}.

\begin{figure}
	\begin{center}
		\includegraphics[width=0.7\linewidth]{Images/EdgeImpulse/Doxygen.png}
		\caption{Doxygen}
		\label{Doxygen}
	\end{center}
\end{figure}

\cleardoublepage
\addcontentsline{toc}{chapter}{\TRANS{Literaturverzeichnis}{Bibliography}}
\printbibliography

\cleardoublepage

%\renewcommand{\indexname}{\TRANS{Stichwortverzeichnis}{Index}}
%\addcontentsline{toc}{chapter}{\TRANS{Stichwortverzeichnis}{Index}}
\printindex

\end{document}

