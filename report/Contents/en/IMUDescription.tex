	\section{IMU Overview}
	\subsection{Introduction}
	The Inertial Measurement Unit (IMU) is an integral component of the Arduino Portenta H7, providing critical data for motion sensing and orientation tracking. The IMU typically consists of a combination of accelerometers, gyroscopes, and sometimes magnetometers. Here's a detailed overview of the IMU in the Arduino Portenta H7:
	
	\subsection{Components of the IMU}
	The IMU in the Arduino Portenta H7 comprises the following key sensors:
	\begin{itemize}
		\item \textbf{Accelerometer}: Measures linear acceleration along three axes (X, Y, and Z). It provides data on how fast the device's velocity is changing.
		\item \textbf{Gyroscope}: Measures angular velocity around the three axes (X, Y, and Z). It helps in tracking the device's orientation and rotational movements.
		\item \textbf{Magnetometer (optional)}: Measures the magnetic field along three axes. It is often used to provide compass heading information and to correct for gyroscopic drift.
	\end{itemize}
	
	\subsection{Functionality and Data Provided}
	
	\subsection{Accelerometer}
	\begin{itemize}
		\item Measures the rate of change of velocity, providing acceleration data in m/s².
		\item Useful for detecting motion, vibration, and tilt.
		\item Can determine the angle of tilt relative to the Earth's gravity.
	\end{itemize}
	
	\subsection{Gyroscope}
	\begin{itemize}
		\item Measures the rate of rotation in degrees per second (°/s) or radians per second (rad/s).
		\item Provides data on how fast and in which direction the device is rotating.
		\item Essential for applications requiring precise control of orientation and stabilization.
	\end{itemize}
	
	\subsection{Magnetometer}
	\begin{itemize}
		\item Measures the strength and direction of magnetic fields.
		\item Often used in conjunction with accelerometer data to provide accurate orientation.
		\item Useful in navigation systems as a digital compass.
	\end{itemize}
	
	
	\subsection{Technical Specifications}
	The specific technical specifications of the IMU in the Arduino Portenta H7 can vary depending on the exact model and manufacturer. However, common specifications include:
	
	\subsection{Accelerometer}
	\begin{itemize}
		\item Measurement Range: ±2g, ±4g, ±8g, ±16g
		\item Sensitivity: Varies by range
		\item Output Data Rate: Up to several kHz
	\end{itemize}
	
	\subsection{Gyroscope}
	\begin{itemize}
		\item Measurement Range: ±125°/s, ±250°/s, ±500°/s, ±1000°/s, ±2000°/s
		\item Sensitivity: Varies by range
		\item Output Data Rate: Up to several kHz
	\end{itemize}
	
	\subsection{Magnetometer (if included)}
	\begin{itemize}
		\item Measurement Range: ±4800µT
		\item Sensitivity: Typically around 0.6µT/LSB
		\item Output Data Rate: Up to 100 Hz
	\end{itemize}
	
	\section{Integration and Usage}
	To effectively use the IMU in the Arduino Portenta H7, one typically follows these steps:
	\begin{enumerate}
		\item \textbf{Initialization}: Configure the IMU sensor settings such as the measurement range and output data rate.
		\item \textbf{Data Acquisition}: Continuously read data from the accelerometer, gyroscope, and magnetometer (if available).
		\item \textbf{Processing}: Apply sensor fusion algorithms to combine data from multiple sensors, enhancing accuracy and reliability.
		\item \textbf{Application}: Use the processed data for the desired application, such as motion tracking, orientation estimation, or gesture recognition.
	\end{enumerate}
	
	
	
	\subsection{Programming Example}
	
	\begin{lstlisting}[language=C++, caption=Arduino IMU Example, label=lst:arduino_imu_example]
		// This example shows how to read IMU data from the Arduino LSM9DS1 sensor
		
		#include <Wire.h>
		#include <Arduino_LSM9DS1.h>
		
		void setup() {
			Serial.begin(9600);
			while (!Serial);
			
			if (!IMU.begin()) {
				Serial.println("Failed to initialize IMU!");
				while (1);
			}
			
			Serial.println("IMU initialized!");
			Serial.print("Accelerometer sample rate = ");
			Serial.print(IMU.accelerationSampleRate());
			Serial.println(" Hz");
		}
		
		void loop() {
			float x, y, z;
			
			// Read acceleration data
			if (IMU.accelerationAvailable()) {
				IMU.readAcceleration(x, y, z);
				Serial.print("Acceleration in G's -> X: ");
				Serial.print(x);
				Serial.print(", Y: ");
				Serial.print(y);
				Serial.print(", Z: ");
				Serial.println(z);
			}
			
			delay(1000);
		}
	\end{lstlisting}
	
	
