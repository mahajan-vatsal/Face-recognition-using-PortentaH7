\chapter{Power LED}

\section{Description}

The Arduino Portenta H7 features a Power LED, an essential component designed to provide a visual indication of the board's power status. This LED is located on the board and serves the following purposes: \cite{arduino_portenta_h7_datasheet:2025}

\begin{itemize}

\item \textbf{Power Status Indicator:} The primary function of the Power LED is to indicate whether the board is receiving power. When the Arduino Portenta H7 is connected to a power source, such as via a USB Type-C cable or a connected battery, the Power LED lights up. This immediate visual feedback confirms that the board is powered on and operational.

\item \textbf{Troubleshooting Aid:} The Power LED is a valuable tool for troubleshooting. If the board does not appear to be functioning correctly and the Power LED is off, this indicates that the board is not receiving power, prompting the user to check the power connections and source.

\item \textbf{Continuous Operation:} The Power LED remains illuminated as long as the board is powered. This continuous indication helps users quickly verify that the board is in a powered state during development, testing, and deployment.

\end{itemize}

\textbf{Key Features of the Power LED:}
\begin{itemize}
	\item \textbf{Visibility:} The Power LED is positioned for easy visibility, ensuring that users can quickly ascertain the power status without needing to connect additional peripherals or interfaces. \cite{arduino_portenta_h7_datasheet:2025}
	\item \textbf{Simplicity:} The Power LED provides a straightforward, unambiguous indication of the board's power state, simplifying the user experience and aiding in efficient board management.
\end{itemize}

The inclusion of the Power LED on the Arduino Portenta H7 enhances its usability, providing clear and immediate feedback on the power status, which is crucial for effective development and troubleshooting.

\section{Specific Sensor}
While labelled as a power LED, the single green LED on the PortentaH7 isn’t solely for power indication. It has multi-functionality depending on board state and user code interaction. It lights steadily green when powered during typical operation, similar to other Arduino models. However, it flashes rapidly during serial communication and bootloader mode. Notably, when the user code enters deep sleep mode, the LED turns off entirely for power saving. This includes power status, communication activity, and sleep mode activation. Understanding these LED behaviours can aid in troubleshooting and code debugging. The green power LED’s primary function indicates power and basic board states. \cite{arduino_portenta_h7_datasheet:2025}

\section{Specification}

\begin{itemize}
	
	\item The power LED on the Arduino Portenta H7 is a green LED.
	\item It is connected to pin 25 on the board.
	\item The brightness of the power LED can be adjusted using \textcolor{red}{Pulse Width Modulation (PWM)} techniques.
	\item The power LED operates in an active-high configuration, meaning it turns on when the pin is set to a high voltage level.
	\item Users can control the power LED programmatically by setting the pin connected to it (pin 25) to either \textcolor{red}{HIGH} or \textcolor{red}{LOW}.
	\item In order to use the power LED, the pin (pin 25) must be defined as an output in the function \textcolor{red}{setup} using \textcolor{red}{\texttt{pinMode(LED\_PWR, OUTPUT)}}. Otherwise, the LED will not function properly.
	\item Additionally, pin 25 can be utilized for other purposes. In such cases, the power LED will only turn on when the board is connected to a power source.
	
\end{itemize}


\section{Simple Code}
In the below sketch ~\ref{TestLEDPower.ino}, a variable is connected to pin 25. The pin 25 is defined as an output in the function \SHELL{setup}. In the function \SHELL{loop}, the LED is switched on for 1 second and switched off for 1 second so that the LED flashes accordingly.


\begin{code}
	\lstinputlisting[language=python]{../Code/Arduino/LED/TestLEDPower/TestLEDPower.ino}

	\caption{Simple sketch to check the battery state using the power LED.}\label{TestLEDPower.ino}
\end{code}


This is just a simple example. The variable \SHELL{LED\_PWR} is already defined, so the assignment is not necessary. The command delay should be avoided in an Arduino sketch. Instead, variables of the type elapsedMillis should be used.

\section{Test}
\subsection{Simple Function Test}
The simplest test is the flashing of the LED at 2 Hz.

\begin{code}
	\lstinputlisting[language=python]{../Code/Arduino/LED/TestLEDPowerBrightness/TestLEDPowerBrightness.ino}
	
	\caption{Simple sketch to check the battery state using the power LED}\label{TestLEDPowerBrightness.ino}
\end{code}

\subsection{Test all Functions}
The brightness of the power LED can be controlled. This is demonstrated in the example sketch 10.3.
Using the pulse width modulation, the brightness is gradually increased to the maxi- mum value and then gradually reduced to 0 again.


\section{Simple Application}
There are different situations where it might be useful to program the power LED of the Arduino Nano. For example, you could use it to:

\begin{itemize}
	\item Indicate the status of the board, such as whether it is connected to a power source, a computer, or a sensor.
	\item Display the battery level of the board, by changing the brightness or color of the power LED.
	\item Create a visual alarm or notification, by making the power LED blink or flash in a certain pattern.
\end{itemize}

A simple application is to check the condition of the battery. The sketch in Listing demonstrates that if the voltage drops too low, the power LED flashes.

\begin{code}
	\lstinputlisting[language=python]{../Code/Arduino/LED/TestLEDPowerBattery/TestLEDPowerBattery.ino}
	
	\caption{Simple sketch to check the battery state using the power LED}\label{TestLEDPowerBattery.ino}
\end{code}