%%%%%%%%%%%%%%%%%%%%%%%%
%
% $Autor: Wings $
% $Datum: 2020-07-24 09:05:07Z $
% $Pfad: GDV/Vortraege/latex - Ausarbeitung/Kapitel/Einleitung.tex $
% $Version: 4732 $
%
%%%%%%%%%%%%%%%%%%%%%%%%

\chapter{Face Recognition System for Access Control}
\section{Introduction}

Aadvancements in technology have transformed smart devices, enabling them to learn and adapt through machine learning (ML) \cite{liu:2018} and edge computing \cite{jiang:2020}. Edge devices allow AI models to operate locally, reducing latency, preserving privacy, and cutting costs—ideal for real-time applications like access control. By handling data directly on devices, edge AI mitigates issues like bandwidth constraints and security concerns, supporting fast and secure decision-making even in low-connectivity environments \cite{datascience_edge_ai:2024} \cite{Research_Gate:2019} \cite{chen:2023}. 
\\

Face recognition is a technology that uses face image of someone to verify his identity by finding this person in a given photos database. It becomes very practical in access control systems as it does not require any physical interaction for gaining access as traditional ways with keys. Moreover, these systems only require a camera for recognition and are easy to install and use. This is why they are already in use by companies as access control to their offices, in home automation systems. \cite{yang:2020}
Face recognition in access control offers a contactless, secure, and efficient solution by authenticating individuals through unique biometric features, delivering convenience and enhanced security \cite{ieee:2021} .

\section{Applications of Face Recognition in Access Control Systems}
Face recognition for access control is gaining popularity across a wide range of industries due to its security, efficiency, and convenience. The technology is applied in various settings, including:

\subsection{Building Access and Security}
Face recognition systems in commercial and residential buildings grant access only to authorized personnel, enhancing security with a contactless approach suitable for offices, healthcare, and residential complexes   \cite{Blog_norden:2024}

\subsection{Smart Home Systems}
In smart homes, face recognition enables personalized, secure access, managing locks and automating entry for family members and approved visitors \cite{ieee:2021}

\subsection{Time and Attendance Tracking}
Face recognition automates attendance in workplaces and schools, replacing badges with accurate, contactless tracking. \cite{attenface:2022}

\subsection{Healthcare and Critical Facilities}
In healthcare, face recognition restricts access to sensitive areas like drug storage, ensuring only authorized staff can access, providing a higher level of security in sensitive environments. \cite{cyberlink:2024}
\\

The use of face recognition in access control systems simplifies user interactions by offering a contactless and efficient way to authenticate and authorize individuals.

\section{Topic Description}

In this project, we aim to develop a face recognition-based access control system using the Arduino Portenta H7 ~\ref{sec:portentaH7} and the Vision Shield with the help of Edge Impulse ~\ref{sec:edge_impulse}. The system is designed to offer a secure, contactless, and efficient method of controlling access to restricted areas, such as offices, homes, or high-security zones.

The core functionality revolves around capturing an individual's face using the Vision Shield’s camera and processing the image using a machine learning model deployed on the Arduino Portenta H7 \cite{arduinoVisionShield:2024}. The model is trained to recognize authorized personnel and differentiate them from unauthorized individuals. Upon successful recognition, the system grants access, such as unlocking a door or enabling entry, while denying it in case of unrecognized faces.
\begin{comment}
	This system is particularly useful in:
	\begin{itemize}
		\item \textbf{Corporate Offices:} Controlling access to different departments or secured rooms.
		\item \textbf{Residential Buildings:} Granting entry to residents while keeping out unauthorized individuals.
		\item \textbf{Data Centers or Server Rooms:} Providing access only to authorized personnel.
	\end{itemize}
\end{comment}

By deploying the solution on an embedded device like Arduino Portenta H7, we enable local image processing, which reduces the need for internet connectivity. Although the Portenta H7 \cite{portentaH7doc:2024} is not a strict real-time device, it performs sufficiently fast for practical access control applications. The system is also scalable, allowing for easy updates to include new faces as needed, making it a flexible and secure solution for access control.

\begin{comment}
	\section{Why Edge Impulse for Face Recognition in Access Control?}
	
	For this project, we chose face recognition as the main technology for access control because of its clear advantages in terms of security, convenience, and user experience. Alongside that, we decided to use \textbf{Edge Impulse} as the platform to deploy this system, given its strong capabilities for edge devices and real-time processing, which are essential for access control systems. Here's why we opted for face recognition and why we selected Edge Impulse instead of other popular platforms like \textbf{Google Colab}, \textbf{TensorFlow} or \textbf{PyTorch}.
	
	
	\subsection{Face Recognition for Access Control}
	Face recognition technology brings several key benefits when used for access control systems:
	
	\begin{itemize}
		\item \textbf{Security}: Face recognition provides a much more secure way of identifying people than traditional methods like passwords or keycards. With biometrics, it's harder for unauthorized individuals to gain access.
		
		\item \textbf{Convenience}: With face recognition, users no longer need to carry keycards or remember passwords. It offers a hands-free, easy access experience—users just show their face.
		
		\item \textbf{Contactless and Hygienic}: In environments where hygiene is a priority, like healthcare or public spaces, contactless entry methods like facial recognition are crucial. No need to touch any devices or surfaces, which helps reduce the spread of germs.
		
		\item \textbf{Real-Time Processing}: Access control systems need to make decisions quickly to avoid delays. Face recognition can operate in real time, processing data instantly to allow or deny access.
	\end{itemize}
	
	Given these advantages, we decided that face recognition would be the best solution for our access control system. To ensure it works effectively on the edge device, we needed a platform that could handle real-time processing efficiently.
	
	
	\subsection{Why Edge Impulse for this Use Case?}
	
	When it comes to running face recognition on devices like the Arduino Portenta H7, Edge Impulse offers several important benefits compared to other platforms such as Google Colab, TensorFlow, PyTorch, or OpenCV:
	
	\begin{itemize}
		\item \textbf{Real-Time Performance}: In access control, fast and accurate face recognition is essential. Unlike cloud-based platforms like Google Colab, which may have delays due to network issues, Edge Impulse runs everything on the device, ensuring decisions are made in real time, without any lag.
		
		\item \textbf{Seamless Embedded Deployment}: Platforms like TensorFlow and PyTorch, while powerful, require extra steps to optimize models for edge devices. Edge Impulse, however, streamlines this process by automatically optimizing models for embedded systems like the Portenta H7. This saves time and ensures that the models are efficient for resource-limited devices.
		
		\item \textbf{Privacy and Data Security}: Since Edge Impulse processes data locally on the device, there’s no need to send sensitive biometric data to the cloud. This is crucial for access control systems, where privacy and security are a top priority. By keeping everything on-device, the risk of data breaches is greatly reduced.
		
		\item \textbf{Quick and Easy to Deploy}: Edge Impulse offers a user-friendly interface that simplifies data collection, model training, and deployment. This reduces the complexity of integrating the system with an edge device, allowing for faster deployment. In contrast, platforms like Google Colab or TensorFlow often require more manual work to set up for edge environments.
		
		\item \textbf{Power Efficiency}: Access control systems need to run continuously, so power efficiency is critical. Edge Impulse is designed to create models that are lightweight and optimized for low-power devices, making it perfect for the Portenta H7 and similar hardware. This ensures the system can run for extended periods without consuming too much power.
		
		
	\end{itemize}
	
	
	\section{Restrictions from Hardware and Environment for Edge Impulse}
	
	\subsection{Limitations}
	
	%\subsubsection{Restriction from Edge Impulse}
	
	
	Although the typical inference latency is 20–50 ms on embedded devices but may rise with complex models, impacting real-time applications requiring faster responses. \cite{ieee:2021ImageProcessing}
	
	In conclusion, face recognition was chosen for its high security and ease of use in access control, and Edge Impulse was selected for its ability to deliver real-time, efficient processing on edge devices. While other platforms provide more flexibility, Edge Impulse's simplicity, privacy features, and seamless integration with embedded systems made it the best choice for this application.
	
	\begin{itemize}
		\item \textbf{Processing Power and Memory Limitations:} The Arduino Portenta H7, while powerful for embedded applications, has limited processing capabilities compared to dedicated servers or large edge devices and only up to 8MB of RAM necessitating the use of lower-resolution images. This means models need to be lightweight and optimized, which can sometimes compromise accuracy, especially in less controlled environments. 
		
		\item \textbf{Storage Constraints:} The onboard storage of the Portenta H7 is limited, which restricts how many face profiles and training images you can keep. This means managing face profiles (adding or removing) needs to be done carefully  to avoid running out of memory.
		
		\item \textbf{Camera Quality and Vision Shield Constraints:} The Vision Shield’s camera doesn’t match the resolution of high-end cameras, which can impact its ability to capture detailed facial features. This is particularly challenging in varying lighting conditions.
		
		\item \textbf{Power Consumption:} Designed for low-power use, the Portenta H7 may face slower response times during resource-intensive tasks, like continuous face recognition, which could affect performance.
		
		\item \textbf{Latency and Real-Time Performance:} Running models in real-time on the Portenta H7 can lead to delays in face identification, especially in crowded settings or when there are multiple faces to process simultaneously.
		
		\item \textbf{Environmental Constraints:} Changes in ambient lighting, such as direct sunlight or shadows, can significantly impact recognition accuracy. Additionally, how the camera is positioned can limit its ability to effectively detect faces.
		
		\item \textbf{Face Dataset and Profile Limits:} The small dataset, consisting of only 3 distinct faces with 50-70 images each, restricts how well the model can generalize. Managing these face profiles might also require off-device processes due to the hardware constraints.
	\end{itemize}
	
\end{comment}

\section{Challenges and Limitations}

\begin{itemize}
	\item \textbf{Processing Power}: The Arduino Portenta H7 provides up to 1 MB of internal RAM and 8 MB of external SDRAM, suitable for constrained models but it's typically not enough for running big networks without pruning, quantization, or other optimizations. \cite{ieee:2021Comparison} \cite{wang:2023}
	
	\item \textbf{Storage Constraints}: With 2 MB of internal Flash and 16 MB of external NOR Flash, the Portenta H7 provides sufficient storage for basic applications. However, tasks like managing large face datasets or multiple user profiles can quickly exceed its capacity, requiring careful memory management.
	
	\item \textbf{Camera Quality and Vision Shield Limitations}: The Vision Shield camera we utilize QQVGA resolution to support upto 120FPS, which may affect face recognition in low or dynamic lighting. \cite{arduinoVisionShield:2024}.
	
	\begin{comment}
		\item \textbf{Power Consumption}: In continuous recognition tasks, the Portenta H7 consumes approximately 140–150 mA per image frame. This power demand can reduce performance in prolonged, resource-intensive use cases, especially for battery-operated systems.
		
	\end{comment}
	\item \textbf{Latency and Performance}: Real-time face recognition on the Portenta H7 can experience delays, particularly with multiple faces in view or in crowded settings, as it processes each image in roughly 20–50 ms. \cite{wang:2023} \cite{ieee:2021ImageProcessing}
	
	\item \textbf{Model customization}: While Edge Impulse offers many advantages, it has limitations compared to platforms like TensorFlow and PyTorch, which provide greater flexibility in model customization. Edge Impulse is focused on simplicity and rapid deployment, leaving less options for advanced tuning. \cite{arxiv:2022}
	
	\item \textbf{Environmental Constrains}: Variability in ambient lighting, such as bright sunlight or shadows, can impact model accuracy. Additionally, camera placement is crucial; poor angles can reduce recognition effectiveness.
	
	\item \textbf{Face Dataset and Profile Capacity}: The limited dataset (3 individuals, 50–70 images each) restricts the model’s ability to generalize well across diverse faces. Profile management is required due to storage constraints, and processing may need to be offloaded periodically.
\end{itemize}

\section{Report Structure}
The first chapter introduces the project, covering its background, applications, topic description and challenges. Chapter two discusses the development tools used, including the Arduino IDE and its web version. Chapter three outlines the hardware specifications of the Arduino Portenta H7, Vision Shield and Cat. M1/NB IoT GNSS Shield.

Chapter four focuses on the onboard sensors of the Portenta H7 and outlines their features and capabilities. The fifth chapter covers the description of major tools and libraries relevant to Machine Learning along with TensorFlow Lite, detailing its integration for efficient model inference and finally with Edge impulse studio which is used in our project. Chapter six provides an overview of machine learning algorithms and packages relevant to the system. Chapter seven introduces the Knowledge Discovery in Databases (KDD) methodology and explains its significance for the project. Finally, chapter eight describes the practical implementation of the KDD process in developing the face recognition system.