
	
	\section{Built-in Push Button}
	
	\subsection{General}
	A push button is a simple switch mechanism used to control various devices and processes. It is typically made of hard materials like plastic or metal. The surface of a push button is designed to be easily depressed or pushed by the human finger or hand. When you press a push button, it either closes or opens an electrical circuit.
	
	In industrial and commercial applications, push buttons can be linked together so that pressing one button releases another. Emergency stop buttons, often with large mushroom-shaped heads, enhance safety in machines and equipment. Pilot lights are sometimes added to push buttons to draw attention and provide feedback when the button is pressed. Color-coding is common to associate push buttons with their specific functions (e.g., red for stopping, green for starting).
	
	\subsection{Specific Sensor}
	The Arduino Nano 33 BLE Sense features an onboard push button. This button is a simple electrical switch that can be activated by pressing it. When you press the button, it completes an electrical circuit. The push button is designed for user interaction and can be used for various purposes.
	
	The built-in button \texttt{BUTTON\_B} is connected with pin 11. Using the function \texttt{pinMode(BUTTON\_PIN, INPUT\_PULLUP)} the pin is declared as an input. As can be seen in the sketch, pressing the button can be used to trigger actions; typical actions include switching on an LED, changing modes, or initiating sensor readings. Overall, the push button provides a convenient way to interact with the Arduino Nano 33 BLE Sense and create responsive projects.
	
	\subsection{Specification}
	\begin{itemize}
		\item The built-in button is a small white button and connected to pin 11.
		\item Built-in Button: \texttt{BUTTON\_B = 11u}
		\item If the pin is declared as input in the function setup, then it can be used.
		\item The pin 11 must be defined as an input in the function setup by setting \texttt{pinMode(11, INPUT\_PULLUP)}, otherwise the button cannot be read.
		\item The pin 11 can also be used otherwise. Then the button is not in use.
	\end{itemize}
	
	\subsection{Bibliothek}
	No special library is required to operate the built-in button.
	
	\subsection{Simple Code}
	As soon as the button is connected, it can be used. It is not necessary to install a special library. Programming takes place in two steps:
	
	\begin{enumerate}
		\item In the first step, the pin is configured in the function setup:
		\begin{verbatim}
			pinMode(BUTTON_B, INPUT_PULLUP)
		\end{verbatim}
		\item In the second step, the button can be used in the function loop. To read in the value, use the function \texttt{digitalRead}:
		\begin{verbatim}
			buttonState = digitalRead(BUTTON_B);
		\end{verbatim}
	\end{enumerate}
	