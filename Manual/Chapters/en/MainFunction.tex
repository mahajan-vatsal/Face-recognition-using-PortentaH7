%%%%%%%%%%%%
%
% $Autor: Wings $
% $Datum: 2019-03-05 08:03:15Z $
% $Pfad: MainFunction.tex $
% $Version: 4250 $
% !TeX spellcheck = en_GB/de_DE
% !TeX encoding = utf8
% !TeX root = manual 
% !TeX TXS-program:bibliography = txs:///biber
%
%%%%%%%%%%%%

\chapter{Main Function}

\section{User Interaction and Workflow}
The Face Recognition for Access Control System is designed to provide a seamless and user-friendly experience. Below are the key functionalities and how users can interact with the system:

\subsection{User Registration}

\subsubsection{Adding a New User:} 
To add a new user, the system administrator or authorized user can use the system’s interface to capture the user’s face image. The user simply needs to capture images from the Vision Shield camera and do the model retraining along with redeployment as mentioned in the report.

\subsubsection{Profile Management:} 
Each user profile can be labeled with a unique identification name for easy management. The system allows administrators to add, update or delete user profiles as needed.

\subsection{Access Control}

\subsubsection{Real-Time Recognition:} 
When a user approaches the access point, the system automatically captures their face image using the Vision Shield camera. The captured image is compared against the registered profiles in real-time.

\subsubsection{Access Granting/Denial:} 
If the system recognizes the user and matches their face with a registered profile, access is granted. If no match is found, access is denied and an alert can be sent to the administrator for further action.

\subsection{System Feedback}

\subsubsection{GUI Interface:} 
The system provides immediate feedback through a graphical user interface (GUI) hosted on the local host. The GUI displays whether the person is identified or not, along with relevant details such as the Person name and the inference time. This interface is accessible to administrators for real-time monitoring and decision-making.

\section{Model details:}
The system leverages the MobileNet V2 algorithm optimized for edge devices like the Portenta H7. This ensures fast and accurate face recognition while maintaining low power consumption. The model is trained using Edge Impulse Studio, which allows for easy customization and retraining based on specific user requirements.