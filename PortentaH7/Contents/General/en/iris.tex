%%%%%%
%
% $Autor: Wings $
% $Datum: 2021-05-14 $
% $Pfad: GitLab/Bilderkennung/Projects $
% $Dateiname: iris
% $Version: 4620 $
%
%%%%%%

%todo Manche Texte finden sich in TensorFlow.tex wieder. Dies ist zu ändern.
%Quelle: https://riptutorial.com/de/scikit-learn/example/6801/beispieldatensatze


\subsection{Fisher's Iris Data Set\index{Dataset!Fisher's Iris Data Set}}

The recognition of the iris is another classic for image classification. There are also several tutorials on this classic. The Fisher's Iris Data Set\index{Dataset!Fisher's Iris Data Set} was used in R.A.~Fisher's classic 1936 work and can be found in the UCI Machine Learning Repository\index{UCI Machine Learning Repository}. \cite{Fisher:1936,Iris:2021,Schutten:2016,Unwin:2021} Four features of the flowers Iris Setosa, Iris Versicolour and Iris Virginica were measured. For each of the three classes 50 data sets with four attributes are available. The width and length of the sepal and petal were measured in centimetres. \cite{Anderson:1935,Sahni:2006}. One species of flower is linearly separable from the other two, but the other two are not linearly separable from each other.


The data set can be accessed in several places:

\begin{itemize}
    \item \href{https://archive.ics.uci.edu/ml/datasets/iris}{Archive of machine learning datasets from the University of California at Irvine.} \cite{UCIIris:2021}
    \item \href{https://scikit-learn.org/stable/auto_examples/datasets/plot_iris_dataset.html}{Python package \FILE{skikitlearn}} \cite{scikit-learn:2011}
    \item \href{https://www.kaggle.com/uciml/iris}{Kaggle - Machine Learning Website} \cite{KaggleIris:2018}
    %\item \url{https://gist.github.com/curran/a08a1080b88344b0c8a7}
\end{itemize}

The data set is available on various websites, but attention must be paid to the structure of the data set. Since it is a file in CSV format, the structure is column-oriented. As a rule, the title of the individual column is given in the first line: \texttt{sepal\_length}, \texttt{sepal\_width}, \texttt{petal\_length}, \texttt{petal\_width} and \texttt{species}. The values are in centimetres, the species is given as \texttt{setosa} for Iris setosa, \texttt{versicolor} for Iris versicolor and \texttt{virginica} for the species Iris virginica.
The columns in this record are:

\begin{itemize}
  \item Sequence number
  \item Length of sepal in centimetres
  \item Sepal width in centimetres
  \item Petal length in centimetres
  \item Petal width in centimetres
  \item Class    
\end{itemize}




The aim is to classify the three different iris species based on the length and width of the sepal and petal. Since the dataset is delivered with the library \PYTHON{sklearn}, this approach is chosen. 

\begin{code}
  \begin{lstlisting}[language=MyPython, numbers=left,label={src:irisimport}]
from sklearn.datasets import load_iris
iris = load_iris()
  \end{lstlisting}
  \caption{Loading the Fisher's Iris Data Set\index{Dataset!Fisher's Iris Data Set}}
\end{code}


The record is a \PYTHON{dictionary}. Its keys can be easily displayed:

\begin{lstlisting}[language=MyPython, numbers=left]
>>> iris.keys()
\end{lstlisting}

The related issue is as follows:

\begin{lstlisting}[numbers=none]
    dict\_keys(['data', 'target', 'frame', 'target\_names', 'DESCR', 'feature\_names', 'filename'])
\end{lstlisting}



The individual elements can now be viewed. After entering the command 

\begin{lstlisting}[language=MyPython, numbers=left]
iris['DESCR']
\end{lstlisting}

a detailed description is output:

\begin{code}
\begin{lstlisting}[language=MyPython, numbers=left]
'.. _iris_dataset:
    
Iris plants dataset
--------------------
    
**Data Set Characteristics:**
    
:Number of Instances: 150 (50 in each of three classes)
:Number of Attributes: 4 numeric, predictive attributes and the class
:Attribute Information:
   - sepal length in cm
   - sepal width in cm
   - petal length in cm
   - petal width in cm
   - class:
       - Iris-Setosa
       - Iris-Versicolour
       - Iris-Virginica
      
   :Summary Statistics:
   
   ============== ==== ==== ======= ===== ====================
   Min  Max   Mean    SD   Class Correlation
   ============== ==== ==== ======= ===== ====================
   epal length:    4.3  7.9   5.84   0.83    0.7826
   sepal width:    2.0  4.4   3.05   0.43   -0.4194
   petal length:   1.0  6.9   3.76   1.76    0.9490  (high!)
   petal width:    0.1  2.5   1.20   0.76    0.9565  (high!)
   ============== ==== ==== ======= ===== ====================
   
   :Missing Attribute Values: None
   :Class Distribution: 33.3% for each of 3 classes.
   :Creator: R.A. Fisher
   :Donor: Michael Marshall (MARSHALL%PLU@io.arc.nasa.gov)
   :Date: July, 1988
   
   The famous Iris database, first used by Sir R.A. Fisher. The dataset is taken
   from Fisher\'s paper. Note that it\'s the same as in R, but not as in the UCI\nMachine Learning Repository, which has two wrong data points.
   
   This is perhaps the best known database to be found in the
   pattern recognition literature.  Fisher\'s paper is a classic in the field and
   is referenced frequently to this day.  (See Duda & Hart, for example.)  The
   data set contains 3 classes of 50 instances each, where each class refers to a\ntype of iris plant.  One class is linearly separable from the other 2; the\nlatter are NOT linearly separable from each other.
   
   .. topic:: References
   
   - Fisher, R.A. "The use of multiple measurements in taxonomic problems"
     Annual Eugenics, 7, Part II, 179-188 (1936); also in "Contributions to
     Mathematical Statistics" (John Wiley, NY, 1950).
   - Duda, R.O., & Hart, P.E. (1973) Pattern Classification and Scene Analysis.
     (Q327.D83) John Wiley & Sons.  ISBN 0-471-22361-1.  See page 218.
   - Dasarathy, B.V. (1980) "Nosing Around the Neighborhood: A New System
     Structure and Classification Rule for Recognition in Partially Exposed
     Environments".  IEEE Transactions on Pattern Analysis and Machine
     Intelligence, Vol. PAMI-2, No. 1, 67-71.
   - Gates, G.W. (1972) "The Reduced Nearest Neighbor Rule".  IEEE Transactions
     on Information Theory, May 1972, 431-433.
   - See also: 1988 MLC Proceedings, 54-64.  Cheeseman et al"s AUTOCLASS II
     conceptual clustering system finds 3 classes in the data.
   - Many, many more ...'
\end{lstlisting}
\caption{Description of Fisher's Iris Data Set\index{Dataset!Fisher's Iris Data Set}}
\end{code}

With the input

\begin{lstlisting}[language=MyPython, numbers=left]
    iris['feature_names']
\end{lstlisting}

you get the names of the attributes:

\begin{lstlisting}[numbers=none]
    ['sepal length (cm)',
    'sepal width (cm)',
    'petal length (cm)',
    'petal width (cm)']
\end{lstlisting}

The names of the flowers that appear after entering

\begin{lstlisting}[language=MyPython, numbers=left]
    iris['target_names']
\end{lstlisting}

are displayed, are

\begin{lstlisting}[numbers=none]
    array(['setosa', 'versicolor', 'virginica'], dtype='<U10')
\end{lstlisting}

For further investigation, the data with the headings are included in a data frame.

\begin{lstlisting}[language=MyPython, numbers=left]
    X = pd.DataFrame(data = iris.data, columns = iris.feature_names)
    print(X.head())
\end{lstlisting}

The command \PYTHON{(X.head())} shows -- as in Figure~\ref{TensorFlowHead} -- the head of the data frame. It is obvious that each data set consists of four values. 

\begin{figure}[H]
    \GRAPHICSC{0.6}{1.0}{TensorFlow/IrisHead}
    \caption{Header lines of  Fisher's Iris Data Set\index{Dataset!Fisher's Iris Data Set}}\label{TensorFlowHead}
\end{figure}

Each record also already contains its classification in the key \PYTHON{target}. In the figure~\ref{TensorFlowHeadType} this is listed for the first data sets.

\begin{lstlisting}[language=MyPython, numbers=left]
    y = pd.DataFrame(data=iris.target, columns = ['irisType'])
    y.head()
\end{lstlisting}

\begin{figure}[H]
    \GRAPHICSC{1.0}{1.0}{TensorFlow/IrisHeadType}
    \caption{Output of the categories of  Fisher's Iris Data Set\index{Dataset!Fisher's Iris Data Set}}\label{TensorFlowHeadType}
\end{figure}

By means of the command 

\begin{lstlisting}[language=MyPython, numbers=left]
    y.irisType.value_counts()
\end{lstlisting}

can be used to determine how many classes are present. The output of the command is shown in the figure \ref{TensorFlowIrisTypes}; it results in 3 classes with the numbers $0$, $1$ and $2$. There are 50 records assigned to each of them.

\begin{figure}[H]
    \GRAPHICSC{1.0}{1.0}{TensorFlow/IrisTypes}
    \caption{Names of the categories in Fisher's Iris Data Set\index{Dataset!Fisher's Iris Data Set}}\label{TensorFlowIrisTypes}
\end{figure}
