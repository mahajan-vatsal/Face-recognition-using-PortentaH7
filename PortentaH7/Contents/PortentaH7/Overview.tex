%%%%%%%%%%%%%%%
%
% $Autor: Wings $
% $Datum: 2020-01-29 07:55:27Z $
% $Pfad: komponenten/Bilderkennung/Produktspezifikation/IntelNCS2/Inhalt/Einleitung.tex $
% $Version: 1785 $
% !TeX spellcheck = en_GB
%
%
%%%%%%%%%%%%%%%

\chapter{Introduction}

{\tiny Quelle: \url{https://www.arduino.cc/pro/tutorials/portenta-h7}}




Arduino Portenta H7 board is a dual core unit which has STM32H7 ARM Cortex®-M7 running at 480Mhz and Cortex®-M4 running at 240Mhz. This means it is capable of reading and executing two instructions at the same time. These two processors run the Mbed OS , which is an embedded real time operating system(RTOS) that is optimized for low-power microcontrollers. The interesting part of this dual core processor is that both the processors can communicate with each other and other peripherals on the board.  This is done by a mechanism called Remote procedure Call(RPC) which helps in calling functions on the other processor. The two processors can run Arduino sketches on top of the Mbed OS,  MicroPython/JavaScript via an interpreter, native Mbed applications and TensorFlow Lite.

The Arduino core sits on top of the Mbed OS and acts as a middleware which allows programs to leverage the Mbed OS APIs for storage, communication, security, and other hardware interfaces.

The portenta H7 also has an on-chip GPU Chrom-ART Accelerator, which makes it able to connect an external monitor to build a dedicated embedded computer. In addition, it also has a wireless module which is able to connect WiFi and Bluetooth simultaneously. The WiFi interface can also be utilized as an access point thereby creating its own WiFi network and allowing other devices to connect to it.
Another interesting feature of portenta H7 is it has two 80-pin high density connectors at the bottom of the board in order to connect to other devices.  It also has a USB type-C port  which is used to power the board, connect to a display and can also be utilized as a USB hub. 
Some of the Industrial applications of the Arduino Portenta h7 include\cite{PortentaH7:2021}

\begin{itemize}
	
	\item High-end industrial machinery
	\item	Laboratory equipment
	\item	Computer vision
	\item	PLCs
	\item	Industry-ready user interfaces
	\item	Robotics controller
	\item	Mission-critical devices
	\item	Dedicated stationary computer
	\item	High-speed booting computation 
\end{itemize}

